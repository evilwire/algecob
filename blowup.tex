\section{Blowups}
\newcommand{\CP}{\mathbb{CP}}
\newcommand{\del}{\partial}

Blowing up varieties has long been the tool of the algebraic
geometer. However, blow-ups are not specific to algebraic geometry.
In the following, we give two perspectives, one geometric and one
algebraic, and we relate the two by showing that the geometric
realization of an abstract blow-up are the same as the geometric
blow-up.

\subsection{Geometric Construction}

The exposition here follows and elaborates on the discussion in
Voisin's and Griffith and Harris. Here, $X$ is a smooth manifold
emitting an almost complex structure. More precisely, there exists
some automorphism $I: TX \to TX$ such that $I^2 = -1$. Let $Z$ be
a submanifold with complex structure.

\begin{definition}
A \emph{local linear system of equations} for $Z$ over $X$ is an
open chart $U$ together with holomorphic functions $f_1,\dots,
f_k$ with independent differentials where $k = \codim_X Z$ such 
that $Z = V(f_1,\dots,f_k)$. 
\end{definition}

\begin{prop}

\end{prop}

\subsection{Algebraic Geometric Construction}

The theory in algebraic geometry parallels the geometric 
construction, and an avid reader can quickly gloss over these
notes and recognize the parallelism. We begin by recalling 
some classic notions in scheme theory.

\begin{definition}
A morphism $f: Z \to X$ is called a \emph{closed immersion} if $\O_X \to 
f_*\O_Z$ is surjective.
\end{definition}

We say that two closed immersions $f: Z \to X$ and $g: Z' \to X$ 
are isomorphic if there exists an isomorphism $h: Z \to Z'$ such 
that
\[
\begin{diagram}
Z       & \rTo & X        \\
\dTo{h} &      & \dEquals \\
Z'      & \rTo & X
\end{diagram}
\]
is commutative.

\begin{definition}
A \emph{closed subscheme} of $X$ is an isomorphism class of 
closed immersions. 
\end{definition}

We will often write ``$Z$ is a closed subscheme of $X$'' to mean 
that $Z \to X$ is a (representative) closed immersion of its class.
In this case, $Z$ is also given by a local system of parameters.

\begin{prop}
Then there is a one-to-one correspondence between closed 
subschemes and sheaf of ideals $\I$. In the case that $X$ is
finite type $k$-scheme, $I$ is coherent sheaf.
\end{prop}

Here, $\I$ plays the role of the local system of equations. In
the case where $Z$ is smooth over $X$, $\I$ is generated by a
regular sequence, length of which is equal to the codimension
of $Z$ in $X$.

This need not be, of course. In fact, one of the earliest employs
of blowups is to smooth out singular locus of varieties. 
In fact, for the following, we make no assumptions about $\I$ or
$X$.
